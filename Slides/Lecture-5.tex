% ===============================
% Worksheet / Manual Template
% ===============================
\documentclass[12pt, a4paper]{article}

% ===============================
% Metadata
% ===============================
\newcommand*{\CourseCode}{MAT211}
\newcommand*{\CourseTitle}{Calculus II}
\newcommand*{\Chapter}{Limit and Continuity}
\newcommand*{\Topics}{Limit, Continuity, Algebra of Limits}

\newcommand{\ReferenceList}{%
\begin{enumerate}[leftmargin=2em, label={[\arabic{*}]}]
	\item THEODORE SHIFRIN, \textit{MULTIVARIABLE MATHEMATICS: Linear Algebra, Multivariable Calculus, and Manifolds}.
\end{enumerate}
}

\newcommand*{\University}{BRAC University}
\newcommand*{\Department}{Department of Mathematics and Natural Sciences}
\newcommand*{\LogoPath}{Brac_University_Logo} % logo filename (no extension)
\newcommand*{\DocDate}{\today}
\newcommand*{\Instructor}{Emon Hossain}
\newcommand*{\Role}{Lecturer, BRAC University, Dhaka-1212}

% ---------- Encoding & Fonts ----------
\usepackage[T1]{fontenc}
\usepackage[utf8]{inputenc}
\usepackage{lmodern}
\usepackage{microtype}
\IfFileExists{qrcode.sty}{\usepackage{qrcode}}{}

% ---------- Icons & Colors ----------
\usepackage{fontawesome5} % icons
\usepackage[dvipsnames, table]{xcolor}
\definecolor{YTred}{RGB}{255,0,0}
\definecolor{BannerLeft}{RGB}{30,64,175}
\definecolor{BannerRight}{RGB}{56,189,248}
\definecolor{Accent}{RGB}{20,120,200}

% ---------- Page Setup ----------
\usepackage[top=0.6in, bottom=1.0in, left=0.7in, right=0.7in]{geometry}
\usepackage{setspace}
\setstretch{1.2}
\setlength{\parindent}{0pt}
\setlength{\parskip}{6pt}
\pagestyle{empty}

% ---------- Math & Graphics ----------
\usepackage{amsmath, amssymb, amsfonts, mathtools, bm}
\usepackage{graphicx}
\usepackage{tikz, pgfplots}
\pgfplotsset{compat=1.18}
\usetikzlibrary{calc}

% ---------- Tables & Lists ----------
\usepackage{array, booktabs, multirow, colortbl, float}
\renewcommand{\arraystretch}{1.25}
\usepackage{enumitem}
\setlist{leftmargin=*,itemsep=2pt}

% ---------- Boxes & Links ----------
\usepackage[most]{tcolorbox}
\tcbuselibrary{breakable,skins}
\usepackage[
	colorlinks=true,
	linkcolor=MidnightBlue,
	citecolor=OliveGreen,
	urlcolor=RoyalBlue
]{hyperref}
\usepackage[nameinlink, noabbrev]{cleveref}

% ===============================
% Cover Page (styled with icons)
% ===============================
\newcommand{\FrontPage}{%
\begin{titlepage}
	\centering

	% --- Gradient banner across the top ---
	\begin{tikzpicture}[remember picture, overlay]
		\node[anchor=north west,inner sep=0pt] at ($(current page.north west)$) {%
		\begin{tikzpicture}\path[shade, left color=BannerLeft, right color=BannerRight] (0,0) rectangle (\paperwidth,1.8cm);\end{tikzpicture} };
	\end{tikzpicture}
	\vspace*{1.0cm}

	% --- Logo & University Name ---
	\includegraphics[width=0.18\textwidth]{\LogoPath}
	\par
	\vspace{6pt}
	{\Large\sffamily\bfseries \University\par} {\large\sffamily \Department\par}

	% --- Title block ---
	\vspace{24pt}
	{\normalsize\sffamily\scshape Lecture on\par}
	\vspace{6pt}
	{\Large\sffamily\bfseries \CourseTitle\ (\CourseCode)\par}

	\vspace{18pt}
	{\Huge\sffamily\bfseries \textcolor{Accent}{\Chapter}\par}
	\vspace{10pt}
	{\Large\sffamily\bfseries \textcolor{teal}{\Topics}\par}
	\vfill
	\vspace{14pt}
	\rule{0.55\textwidth}{0.8pt}\par
	\vspace{10pt}
	\vfill
	{\large\sffamily\scshape \DocDate\par}
	\vfill
	\vspace{10pt}
	{\large\sffamily\scshape Conducted by\par}
	{\LARGE\sffamily\bfseries \Instructor\par} {\large\sffamily \Role\par}

	% --- Contact / Links with icons ---
	\vspace{18pt}
	\begin{tcolorbox}
		[ colback=gray!3!white, colframe=gray!40!black, boxrule=0.6pt, arc=2pt, left=8pt,right=8pt,top=8pt,bottom=8pt,
		width=0.82\textwidth, enhanced ]
		\centering
		\sffamily {\large \faEnvelope\;\href{mailto:emon.hossain@bracu.ac.bd}{emon.hossain@bracu.ac.bd} \quad\textbar\quad \faEnvelope\;\href{mailto:mdemon7475@gmail.com}{mdemon7475@gmail.com} \\[4pt] For updates subscribe on \quad \textcolor{YTred}{\faYoutube}\; \href{https://www.youtube.com/@emonhossainraihan}{@emonhossainraihan} }
	\end{tcolorbox}

	\vspace{10pt}
	\begin{tikzpicture}[remember picture, overlay]
		\node[anchor=south west,inner sep=0pt] at ($(current page.south west)$) {%
		\begin{tikzpicture}\path[shade, left color=BannerRight, right color=BannerLeft] (0,0) rectangle (\paperwidth,0.9cm);\end{tikzpicture} };
	\end{tikzpicture}
\end{titlepage}%
}

% ===============================
% End Page
% ===============================
\newcommand{\EndPage}{%
\clearpage
\thispagestyle{empty}
\begin{tikzpicture}[remember picture, overlay]
	\node[anchor=north west,inner sep=0pt] at ($(current page.north west)$) {%
	\begin{tikzpicture}\path[shade, left color=BannerLeft, right color=BannerRight] (0,0) rectangle (\paperwidth,1.2cm);\end{tikzpicture} };
\end{tikzpicture}

\vspace*{0.8cm}
\begin{center}
	{\fontsize{38}{42}\selectfont\sffamily\bfseries Thank You!\par}
	\vspace{0.4em}
	{\large\sffamily We’d love your questions and feedback.\par}
	\vspace{1.2em}
	\rule{0.5\textwidth}{0.8pt}\par
	\vspace{1.2em}

	\begin{tcolorbox}
		[ colback=gray!2!white, colframe=gray!35!black, boxrule=0.7pt, arc=3pt,
		width=0.86\textwidth, left=10pt,right=10pt,top=10pt,bottom=10pt, enhanced ]
		\centering
		\sffamily {\fontsize{18}{22}\selectfont\bfseries \Instructor\par}
		\vspace{2pt}
		{\normalsize \Role\par}
		\vspace{1.2em}
		{\large \faEnvelope\ \href{mailto:emon.hossain@bracu.ac.bd}{emon.hossain@bracu.ac.bd}\quad \textbar\quad \faEnvelope\ \href{mailto:mdemon7475@gmail.com}{mdemon7475@gmail.com} \par}
		\vspace{0.8em}
		\begin{minipage}{0.62\textwidth}
			\centering
			{\Large\bfseries \textcolor{YTred}{\faYoutube}\; \href{https://www.youtube.com/@emonhossainraihan}{@emonhossainraihan}}
			\par {\small (Lectures, walkthroughs, and course updates)}
		\end{minipage}%
		\hfill
		\begin{minipage}{0.30\textwidth}
			\centering
			\IfFileExists{qrcode.sty}{ \qrcode[height=3cm]{https://www.youtube.com/@emonhossainraihan} \par\vspace{2pt}{\scriptsize Scan for the channel} }{ \vspace{2pt}\textcolor{YTred}{\faYoutube}\par }
		\end{minipage}
	\end{tcolorbox}

	\vspace{1.6em}
	\rule{0.64\textwidth}{0.8pt}\par
	\vspace{1.2em}

	\begin{tcolorbox}
		[ colback=white, colframe=Accent, boxrule=0.8pt, arc=3pt, width=0.86\textwidth,
		left=12pt,right=12pt,top=10pt,bottom=10pt, enhanced, title=\sffamily\bfseries\large
		\faBook\ References ] \sffamily\large \ReferenceList
	\end{tcolorbox}
\end{center}

\begin{tikzpicture}[remember picture, overlay]
	\node[anchor=south west,inner sep=0pt] at ($(current page.south west)$) {%
	\begin{tikzpicture}\path[shade, left color=BannerRight, right color=BannerLeft] (0,0) rectangle (\paperwidth,0.9cm);\end{tikzpicture} };
\end{tikzpicture}
\clearpage
}

% ===============================
% Modern Box Styles (with neat tabbed titles)
% ===============================

% --- Definition (deep academic blue) ---
\definecolor{DefBack}{RGB}{228,238,250}   % soft bluish background
\definecolor{DefMain}{RGB}{0,90,190}      % deep but calm academic blue

% --- Theorem (balanced emerald green) ---
\definecolor{ThmBack}{RGB}{230,248,235}   % gentle greenish background
\definecolor{ThmMain}{RGB}{0,155,70}      % balanced rich green

% --- Problem (refined amber tone) ---
\definecolor{ProbBack}{RGB}{255,245,232}  % warm ivory-peach background
\definecolor{ProbMain}{RGB}{195,90,0}     % deep amber / rust-orange

% --- Headers (coordinated with Definition) ---
\definecolor{HeadL}{RGB}{20,55,120}       % deep navy (slightly cooler, academic tone)
\definecolor{HeadR}{RGB}{45,135,215}      % calm mid-sky blue (not too bright)

\tcbset{ parthoBase/.style={ breakable, enhanced jigsaw, boxrule=0.6pt, arc=4pt, left=10pt, right=10pt, top=10pt, bottom=10pt, coltitle=black, fonttitle=\bfseries\sffamily, fontupper=\sffamily, drop fuzzy shadow, attach boxed title to top left={yshift=-0.5mm}, % tab overlaps border neatly
boxed title style={ boxrule=0pt, arc=3pt, top=2pt, bottom=2pt, left=8pt, right=8pt, interior style={left color=tcbcolframe!10!white, right color=tcbcolframe!30!white} }, }, }

% --- Definition Box ---
\NewTColorBox{definition}{ O{Definition} }{%
parthoBase, colback=DefBack, colframe=DefMain, title={\faBook\ \ #1}, borderline west={2.5pt}{0pt}{DefMain}, overlay unbroken and first={ \node[opacity=0.06,anchor=north east,inner sep=6pt] at (frame.north east) {\scalebox{2}{\faBookOpen}}; }, }

% --- Theorem Box ---
\NewTColorBox{theorem}{ O{Theorem} }{%
parthoBase, colback=ThmBack, colframe=ThmMain, title={\faLightbulb[regular]\ \ #1}, borderline west={2.5pt}{0pt}{ThmMain}, overlay unbroken and first={ \node[opacity=0.06,anchor=north east,inner sep=6pt] at (frame.north east) {\scalebox{2}{\faLightbulb}}; }, }

\NewTColorBox{proposition}{ O{Proposition} }{%
parthoBase, colback=ThmBack, colframe=ThmMain, title={\faLightbulb[regular]\ \ #1}, borderline west={2.5pt}{0pt}{ThmMain}, overlay unbroken and first={ \node[opacity=0.06,anchor=north east,inner sep=6pt] at (frame.north east) {\scalebox{2}{\faLightbulb}}; }, }


% --- Problem Box ---
\NewTColorBox{problem}{ O{Problem} }{%
parthoBase, colback=ProbBack, colframe=ProbMain, title={\faQuestionCircle[regular]\ \ #1}, borderline west={2.5pt}{0pt}{ProbMain}, overlay unbroken and first={ \node[opacity=0.07,anchor=north east,inner sep=6pt] at (frame.north east) {\scalebox{2}{\faQuestionCircle}}; }, }

% --- Heading Box ---
\NewTColorBox{heading}{ O{Heading} }{%
enhanced, frame empty, interior style={left color=HeadL!95!black, right color=HeadR!90!white}, boxsep=1pt, top=10pt, bottom=10pt, left=10pt, right=10pt, coltext=white, fontupper=\sffamily\bfseries\Large }

% ==========================================
% Document
% ==========================================
\begin{document}
	\FrontPage
	\Large

    \begin{heading}
		Definition of Limit
	\end{heading}

    \vspace{10pt}

    \begin{definition}[Limit]
    Let $U \subset \mathbb{R}^n$ be an open subset containing a neighborhood of
    $a \in \mathbb{R}^n$, except perhaps for the point $a$ itself.
    Suppose $f:U \to \mathbb{R}^m$. We say that
    \[
        \lim_{x\to a} f(x)=\ell
    \]
    ($f(x)$ approaches $\ell \in \mathbb{R}^m$ as $x$ approaches $a$) if for every
    $\varepsilon>0$ there is $\delta>0$ so that
    \[
        \|f(x)-\ell\|<\varepsilon
        \qquad\text{whenever}\qquad
        0<\|x-a\|<\delta.
    \]
    \end{definition}

    \clearpage

    \begin{heading}
        Component wise Limit
    \end{heading}

    \begin{proposition}
        $$\lim_{x\to a} f(x)=\ell$$ if and only if
$$\lim_{x\to a} f_j(x)=\ell_j$$ for all $j=1,\ldots,m$.
    \end{proposition}

    \clearpage
    \begin{heading}
        Examples of Limits
    \end{heading}
    \begin{problem}
    Fix a non-zero vector $b\in\mathbb R^n$. Let $f:\mathbb R^n\rightarrow\mathbb R$ be defined by $f(x)=b\cdot x$. Show that $$\lim_{x\to a} f(x)=b\cdot a$$
    \end{problem}
 
    \clearpage
    \begin{heading}
        Examples of Limits
    \end{heading}
    \begin{problem}
        Let $f:\mathbb R^n\rightarrow\mathbb R$ be defined by $f(x)=\|x\|^2$. Show that $$\lim_{x\to a} f(x)=\|a\|^2$$
    \end{problem}
    \clearpage
    \begin{heading}
        Examples of Limits
    \end{heading}
    \begin{problem}
        Let $f:\mathbb R^2\setminus\{0\}\rightarrow\mathbb R$ be defined by $f\left(\begin{bmatrix}
            x\\y
        \end{bmatrix}\right)=\frac{x^2y}{x^2+y^2}$. Does $\lim_{x\to 0} f(x)$ exist?
    \end{problem}   
    \clearpage
    \begin{heading}
        Examples of Limits
    \end{heading}
    \begin{problem}
        Let $f:\mathbb R^2\setminus\{0\}\rightarrow\mathbb R$ be defined by $f\left(\begin{bmatrix}
            x\\y
        \end{bmatrix}\right)=\frac{x^2}{x^4+y^2}$. Does $\lim_{x\to 0} f(x)$ exist?
    \end{problem} 
    \clearpage
    \begin{heading}
        Examples of Limits
    \end{heading}    
    \begin{problem}
        Let $f:\mathbb R^2\setminus\{0\}\rightarrow\mathbb R$ be defined by $f\left(\begin{bmatrix}
            x\\y
        \end{bmatrix}\right)=\frac{xy}{x^4+y^2}$. Does $\lim_{x\to 0} f(x)$ exist?
    \end{problem} 
    \clearpage
    \begin{heading}
        Algebra of Limits
    \end{heading}
    \begin{theorem}
        Suppose $f$ and $g$ map a neighborhood of $a\in\mathbb{R}^n$ to $\mathbb{R}^m$ and $k$ maps the same neighborhood to $\mathbb{R}$. Suppose $\lim_{x\to a} f(x)=\ell$, $\lim_{x\to a} g(x)=m$ and $\lim_{x\to a} k(x)=c$, then
        \begin{itemize}
            \item $\lim_{x\to a} (f(x)+g(x))=\ell+m$,
            \item $\lim_{x\to a} (f(x)\cdot g(x))=\ell\cdot m$,
            \item $\lim_{x\to a} (k(x)f(x))=c\ell$.
        \end{itemize}
    \end{theorem}
	    \clearpage
    \begin{heading}
            Continuity
    \end{heading}
    \begin{definition}
        Let $U\subset\mathbb R^n$ be an open set containing a neighborhood of $a\in\mathbb R^n$, and let $f:U\rightarrow\mathbb R^m$. We say that $f$ is continuous at $a$ if $$\lim_{x\to a} f(x)=f(a).$$
        That is, $f$ is continuous at $a$ if for every $\varepsilon>0$, there is $\delta>0$ such that $$\|f(x)-f(a)\|<\varepsilon\qquad\text{whenever}\qquad \|x-a\|<\delta.$$
        We say that $f$ is continuous if it is continuous at every point in its domain.
    \end{definition}
    \clearpage
    \begin{heading}
        Continuity and Open Sets
    \end{heading}    
    \begin{proposition}
        Let $U\subset\mathbb R^n$ be an open set. The function $f:U\rightarrow\mathbb R^m$ is continuous at $a\in U$ if and only if for every open set $V\subset\mathbb R^m$, $f^{-1}(V)$ is open (the preimage of an open set is open).
    \end{proposition}
    \clearpage
    \begin{heading}
        Composition of Functions through open sets
    \end{heading}
    \begin{proposition}
        Suppose $U \subset \mathbb{R}^n$ and $W \subset \mathbb{R}^p$ are open,
        functions $f\circ g$ is defined (i.e., $g(x)\in U$ for all $x\in W$).
        $f:U \to \mathbb{R}^m$, $g:W \to \mathbb{R}^n$, and the composition of
        Then if $f$ and $g$ are continuous, so is $f\circ g$.
    \end{proposition}
    Now, Consider the first example. 
    \clearpage
    \begin{heading}
        Continuity and Sequences
    \end{heading}
    \begin{proposition}
        Suppose $U \subset \mathbb{R}^n$ is open and $f:U \to \mathbb{R}^m$.
        Then $f$ is continuous at $a$ if and only if for every sequence $\{x_k\}$
        of points in $U$ converging to $a$ the sequence $\{f(x_k)\}$ converges to $f(a)$.
    \end{proposition}
    \clearpage
    \begin{heading}
        Continuity and Level Sets
    \end{heading}
    \begin{proposition}
        Suppose $f:\mathbb{R}^n \to \mathbb{R}^m$ is continuous. Then for any
        $c \in \mathbb{R}^m$, the level set
        \[
            f^{-1}(\{c\})=\{x\in \mathbb{R}^n : f(x)=c\}
        \]
        is a closed set.
    \end{proposition}
    \EndPage
\end{document}