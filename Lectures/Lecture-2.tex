\newpage
\section{Lecture-02}

\begin{definition}[Euclidean Vector Space]
	A Euclidean vector space is a finite-dimensional inner product space over $\R$.
\end{definition}

\begin{definition}[Inner product space]
	An inner product space is a vector space $V$ over the field $F$ together with an inner product, that is, a map
	\[ \langle \cdot,\cdot \rangle : V \times V \to F\]
	that satisfies the following three properties for all vectors $x,y,z \in V$ and all scalars $a,b \in F$.
	\begin{enumerate}
		\item (Symmetry) $\langle x,y \rangle = \langle y,x \rangle$.
		\item (Linearity in the first argument)  $\langle ax + by,z \rangle = a \langle x,z \rangle + b \langle y,z \rangle$.

		\item (Positive-definiteness) If $x \neq 0$ then,  $\langle x,x \rangle  > 0$.
	\end{enumerate}
\end{definition}

\begin{example}
	Verify that the dot product function $\langle \cdot, \cdot \rangle : \R \times \R \to \R$ is an inner product.

	To verify this we must prove the three conditions shown above. Let $u = (a,b)$ and  $v = (c,d)$. Then,
	\begin{enumerate}
		\item $\langle u, v \rangle = ac + bd = ca + db = \langle v,u \rangle.$
		\item Let $a,b \in \R$ and $u,v,w \in \R^2$ with $u = (u_1,u_2), v = (v_1,v_2), w = (w_1,w_2)$. Then

	\end{enumerate}
	\begin{align*}
		\langle au + bv, w \rangle & = (au_1 + bv_1) w_1 + (au_2+bv_2) w_2            \\
		                           & = au_1w_1 + bv_1w_1 + au_2w_2 + bv_2w_2          \\
		                           & = a(u_1w_1 + u_2w_2) + b(v_1w_1 + v_2w_2)        \\
		                           & = a \langle u,w \rangle + b \langle v,w \rangle.
	\end{align*}
	\begin{enumerate}
		\item[3.] For any $u = (u_1,u_2)$,
		      \[ \langle u, u \rangle = {u_1}^2 + {u_2}^2.\]
		      Since squares of non-zero numbers are positive, we have  ${u_1}^2 >0, {u_2}^2 > 0$ and therefore,  $\langle u,u \rangle > 0$.
	\end{enumerate}
\end{example}

\begin{definition}[Informal]
	Define \textbf{norm} of a vector $x \in V$ as
	\[ || x || = \sqrt{\langle x,x \rangle}.\]
\end{definition}

\begin{definition}[Metric Space]
	A metric space is an ordered pair $(M,d)$ where $M$ is a set and $d$ is a \textbf{metric} (distance) on $M$, i.e., a function
	\[d : M \times M \to \R\]
	satisfying the following axioms for all points $x,y,z \in M$ :
	\begin{enumerate}
		\item $d(x,y) \geq 0$;
		\item $d(x,y) = 0 \iff x = y$ ;
		\item $d(x,y) = d(y,x)$.
		\item $d(x,z) \leq d(x,y) + d(y,z)$.
	\end{enumerate}
\end{definition}

\begin{example}[Tax-cab Metric]
	In $\R^2$, the taxi-cab distance between $p=(p_1,p_2)$ and $q=(q_1,q_2)$ is
	$|q_1 - p_1| + |q_2 - p_2|$. It is easy to verify that this is a metric.
\end{example}

\begin{definition}[Open ball]
	For any $x \in \R^n$ and any $r > 0$, the \textbf{open ball} of radius $r$ around $x$ is the subset
	\[B_r(x) = \{y \in \R^n \mid |x-y| < r\}.\]
\end{definition}

\begin{center}
	\includegraphics[scale=0.18]{Lectures/images/open-set.jpg}
\end{center}

\begin{definition}[Open sets of $\R^n$ ]
	A subset $U \subseteq \R^n$ is open if for every point $x \in U$, there exists $r > 0$ such that the open ball $B_r(x)$ is contained in $U$.
\end{definition}

\begin{example}
	We prove that the Cartesian product \((a_1,b_1)\times(a_2,b_2)\) of two open intervals in \(\mathbb{R}\) is an open set in \(\mathbb{R}^2\) with the standard Euclidean topology.

	\begin{proof}
		Let \(d\) denote the Euclidean metric on \(\mathbb{R}^2\):
		\[
			d\big((x_1,y_1),(x_2,y_2)\big)=\sqrt{(x_1-x_2)^2+(y_1-y_2)^2}.
		\]
		Take an arbitrary point \(p=(x,y)\in (a_1,b_1)\times(a_2,b_2)\). Then
		\[
			a_1 < x < b_1,\qquad a_2 < y < b_2.
		\]
		Define the four positive distances to the endpoints:
		\[
			d_1 = x - a_1,\quad d_2 = b_1 - x,\quad d_3 = y - a_2,\quad d_4 = b_2 - y.
		\]
		Let
		\[
			r = \min\{d_1,d_2,d_3,d_4\}>0.
		\]
		We claim that the open ball \(B(p,r)=\{q\in\mathbb{R}^2 : d(p,q)<r\}\) is contained in \((a_1,b_1)\times(a_2,b_2)\).

		Indeed, take any point \(q=(u,v)\in B(p,r)\). Then
		\[
			|u-x|\le d(p,q)<r,\qquad |v-y|\le d(p,q)<r.
		\]
		Hence
		\[
			x-r < u < x+r,\qquad y-r < v < y+r.
		\]
		Because \(r\le x-a_1\) and \(r\le b_1-x\), we have
		\[
			a_1 \le x-r < u < x+r \le b_1,
		\]
		so \(u\in(a_1,b_1)\). Similarly, \(r\le y-a_2\) and \(r\le b_2-y\) imply
		\[
			a_2 \le y-r < v < y+r \le b_2,
		\]
		so \(v\in(a_2,b_2)\). Consequently \(q=(u,v)\in(a_1,b_1)\times(a_2,b_2)\).

		Thus every point of the product has an open neighbourhood (an open ball) contained in the product. By definition, \((a_1,b_1)\times(a_2,b_2)\) is open in \(\mathbb{R}^2\).
	\end{proof}
\end{example}

\begin{definition}[Closed set]
	A closed set $A$ is defined by its ability to contain all of its limit points, meaning if a sequence of points $\{x_n\}$ within $A$ converges to a limit $x$, then that limit $x$ must also be in $A$.
\end{definition}

\begin{definition}[Sequence]
	A sequence is a function whose domain is $\N$.
\end{definition}
